\section{Data layout and notation}

  To efficiently use FMA instructions we generalize the data layout as
  proposed in~\cite{jeffers2016knl, cpu-myth} to the $N$--dimensional
  case.  {\bf We encourage the reader to get a good understanding of
    our data layout before proceeding to the next section.}

  All the tensors are stored as row--major multi--dimensional, packed
  and properly aligned arrays in memory.  To distinguish memory arrays
  from tensors we will use lower letters to refer to them, and C++
  like syntax to refer to a particular element.

  {\bf Image tensors} are stored as 6--dimensional arrays in memory.
  An image tensor of size $B \times F \times X \times Y \times Z$ is
  stored as an array of size $B \times \ceil{F/S} \times X \times Y
  \times Z \times S$.  Here $S$ is the size of the CPU's vector
  register in floats (16 for AVX512, 8 for AVX, AVX2, and 4 for SSE4).
  A tensor element $I(b,f,x,y,z)$ corresponds to the array element
  $i\sqb{b}\sqb{\floor{f/S}}\sqb{x}\sqb{y}\sqb{z}\sqb{f\mod S}$

  We maintain two copies of a {\bf kernel tensor} in memory -- a
  regular and reflected form.  For a kernel tensor of size $F' \times
  F \times K_x \times K_y \times K_z$ both memory arrays have size
  $\ceil{F'/S} \times \ceil{F/S} \times K_X \times K_Y \times K_Z
  \times S \times S$.  A tensor element $W(f',f,k_x,k_y,k_z)$ is
  stored in $w\sqb{\floor{f'/S}}
  \sqb{\floor{f/S}}\sqb{k_x}\sqb{k_y}\sqb{k_z}\sqb{f\mod S}\sqb{f'
    \mod S}$ in one of the arrays, and in $\bar{w}\sqb{\floor{f'/S}}
  \sqb{\floor{f/S}}\sqb{K_x-k_x}\sqb{K_y-k_y}\sqb{K_z-k_z}\sqb{f'\mod
    S}\sqb{f\mod S}$

  Whenever a kernel tensor is updated, both representations in
  memory are modified.  This introduces negligible overhead, as
  updating kernel tensors counts as a minuscule part of the total
  computation.  Having two copies, however, allows us to reuse the
  same algorithm for the forward and backward phases.

  The kernel update tensor ($\Delta W$) of Eq. (\ref{eq:update}) is
  stored only in the regular form.

  Locating a memory array element in the linear address space is done
  by providing ``strides'' for each dimension and the location of the
  origin (address element at $[0][0]\dots[0]$).  A stride represents
  the linear distance in memory between two adjacent element along
  that dimension.  Given the strides, one can easily represent a
  sub--array by providing a new origin -- zeroth element of the
  sub--array.

  We will use a term sub--tensor to refer to a subset of a tensor with
  the same number or less dimensions.  The memory representation of a
  sub--tensor is then just represented as an additional pointer to the
  new origin inside the memory array representing the original array.
  Thus, any mention of sub--tensors and sub--arrays doesn't imply any
  memory copying.
