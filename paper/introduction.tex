\section{Introduction}

  In a popular benchmarking ``of all publicly accessible
  implementations of convnets,''
  (https://github.com/soumith/convnet-benchmarks) all performance
  numbers are for GPUs, with none given for CPUs.  The popularity of
  GPUs for deep learning has resulted from both hardware and software
  advantages.  The hardware gap has narrowed recently, as the Knights
  Landing (KNL) Xeon Phi CPU delivers more than 6 TFLOPS \cite{},
  compared to the 11 TFLOPS rating of the Titan X Pascal GPU (all
  FLOPS ratings are given for single precision arithmetic).

  There is also a software gap.  All the major deep learning
  frameworks use the cuDNN library, which has been developed and
  refined for several years and is already at version 5.  cuDNN
  achieves close to 100\% utilization of available FLOPS, at least for
  some select convolutional network (ConvNet) architectures (cite).

  This paper focuses on CPU implementation of ConvNets that makes
  efficient use of available FLOPS.  A ConvNet contains convolutional
  and other kinds of layers.  A convolutional layer requires an amount
  of computation that grows with the square of the number of channels
  (images), whereas other kinds of layers scale linearly.  Therefore
  almost all of the computation in a typical ConvNet resides in the
  convolutional layers, which are the subject of the current paper.

  One approach is to optimize code by hand (cite). This can achieve
  the highest efficiency of all approaches, but is time-consuming and
  the speed-up does not generalize beyond the specific architecture.
  In fact, many GPU and CPU deep learning implementations appear to
  contain hand-optimized code for standard benchmark network
  architectures, for which they achieve markedly higher efficiency
  than is typical (cite evidence)

  An alternative approach is to implement convolution using highly
  efficient matrix multiplication libraries.  This provides speedup
  for diverse ConvNet architectures, and has been applied to both
  GPUs~\cite{chetlur2014cudnn,neonnervana} and CPUs
  ~\cite{hadjis2015shallow}.  But the efficiency of the approach
  suffers because of computational and memory overhead due to the
  requirement of creating in--memory matrices.

  Intel has introduced a JIT layer compilation approach in its MKL-DNN
  package.  Given a ConvNet architecture, the MKL-DNN software
  generates optimized assembly code at run time.  The current package
  is for Broadwell and Haswell CPUs, and it is unclear how easily it
  generalizes to other CPUs including KNL.

  We propose a metaprogramming approach that aims to be general with
  respect to both the ConvNet architecture and CPU.  We use C++
  templates to create primitives at compile time.  Given the
  specifications of the CPU, our meta--algorithms optimize and
  vectorize single-threaded code, as well as generate static
  scheduling for parallelizing the computation over multiple cores.
  Our approach tries to achieve generality by leveraging the power of
  the compiler.

  We show that our approach can utilize high percentage of the
  available FLOPS regardless of the CPU generation, as well as have
  near linear scalability with the number of cores.  We are
  competitive with state-of-the-art hand-optimized code for specific
  architectures.  Our code yields better performances on KNL
  compared to the state of the art GPU implementations for 3D
  convolutional networks, even though KNL has less FLOPS/s available.

  We further demonstrate the power of our approach for inference only
  computation.

%  Convolutional layers of a ConvNet are computationally dominating
%  both training and inference.  The forward and backward pass involve
%  convolution of an image with a small kernel.  Typically ``valid''
%  dense convolution, possibly with strides is performed.  More
%  advanced types of convolutions can decomposed to a series of dense
%  convolutions~\cite{szegedy2015going}.

  Even though we provide an efficient, publicly available, open source
  implementation for popular ConvNet primitives, the goal of this
  paper is to point to the right direction for solving the problem,
  rather than giving the final ``optimized'' implementation.

  Efficient algorithms for training and inference of ConvNets with
  larger kernels have been introduced in
  ~\cite{zlateski2016znn,zlateski2016znni}, however the evidence
  suggests that ConvNets with smaller kernels, and larger number of
  layers perform better on nearly all problems. (Cite).

  %% \subsection{GPUs vs CPUs}

  %% GPUs have streaming multiprocessors, while the CPUs have in--order,
  %% memory coherent cores.

  %% GPUs model is so called SIMT (single instruction multiple
  %% threads)... dwell on it..

  %% CPUs on the other side have a complex structure of many cores and
  %% virtual cores, many cache layers and each core capable of performing
  %% SIMD instructions on up to 16 floating point numbers.

  %% In this paper we are not trying to answer the GPU vs CPU question,
  %% but rather propose an algorithm and implementation for ConvNet
  %% primitives that can utilize very high percentage of the theoretical
  %% peak performance of both many--core (Xeon Phi) and multi--core
  %% (Xeon) CPUs.  As shown later, our algorithm performs well on
  %% multi--chip systems as well.

  %% \subsection{Mulit--core vs Many--core}

  %% Explain the differences NUMA, caches, etc..


  \aleks{Summarize and cite the text below}
  A popular convolution implementation is to unroll the data until the
  computation is in the form of a large matrix multiplication
  (Chellapilla et al. (2006)). This is the strategy followed by many
  implementors, since matrix multiplication is a well-tuned linear
  algebra primitive available on virtually any platform. While it is
  possible to provide instances of direct calculation that are faster
  than matrix unrolling (e.g., for large S, Krizhevsky (2014)), it is
  challenging to provide an implementation that is faster for more
  than just a small subset of possible convolution problems.


  %% http://legion.stanford.edu/pdfs/bootcamp2014/10_metaprogramming.pdf
